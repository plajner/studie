\chapter{Computerized Adaptive Testing}
This chapter introduces the concept of Computerized Adaptive Testing (CAT) and summarizes its advantages and disadvantages.
 
CAT is a concept of testing which is getting large scientific attention for about two decades~\cite{Linden2000, Wainer1990, VanderLinden2010}. With CAT we build computer administered and computer controlled tests. The computer system is selecting questions for a student taking the test and evaluating his/her performance. 


The process can be divided into two phases: model creation and testing. In the first one the model of the student is created while in the second one the model is used to actually test examinees. There are many different model types which can be used for adaptive testing. In this work we are going to cover Item Response Theory (IRT), which is a model regularly used for CAT, Bayesian networks, and Neural Networks, which are both models commonly used in many areas of artificial intelligence for a large variety of tasks. We will pay closer attention to these models later on but regardless of the model we choose the testing part follows always the same scheme. With the prepared and calibrated model, CAT testing repeats following steps.

\label{sec:CATprocess}
\begin{itemize}
	\item The next question to be asked is selected.
	\item This question is asked and an answer is obtained.
	\item This answer is inserted into the model.
	\item The model (i.e., our estimation about the student's skill) is updated.
	\item (optional) Answers to remaining question are estimated
\end{itemize}

This procedure is repeated until we reach a termination criterion. There are many different stopping criteria. It can be a time restriction, the number of questions, or a confidence interval of the estimated variables (i.e. reliability of the test).

\section{Advantages of CAT}
\emph{Shorter tests:} One of the most obvious advantages of CAT is that the overall length of a test is reduced. Because questions are selected according to the level of the student he/she is not forced to answer questions which are too easy or too hard. This means the test aims better at discovering the level of the student. That results in the reduction of the length of the test in both time and the number of questions. Usually it is enough to ask as few as half the questions to obtain reliable results. 

\emph{Fairness:} A test in the classical theory usually expects a Gaussian score distribution among the population of student. This expectation yields frequencies of question difficulties to be of the same distribution (most questions are medium difficulty and less of them are hard or easy). Because of that a precision of the resulting score is the best for mediocre students while it drops for students on edges of the scale. CAT on the other hand selects appropriate questions based on the skill of the student. That results in the same precision for each student nevertheless his/her position on the score scale.
Intelligent tutoring system: It is quite easy to convert a CAT test to an intelligent tutoring system. ITS is a system which is designed to uncover student’s weak spots and offer more exercises and materials to learn from.

\emph{Motivation: }While testing a student with a CAT system the optimal probability of successful answer to a question is 50\% (at least while using the IRT student model). Even though a question with such probability may not exists to be selected in every step of the testing it should not get far from this value if the question bank is well designed. This helps to keep a student interested in the test. Weaker students will not get overwhelmed by many difficult questions while a good student will not get bored by easy ones.

\emph{Reseating the exam: }With CAT it is extremely easy to resit the exam (provided we keep track of previous questions for the particular student). Because of its nature CAT system can create a completely different test to retest the same student.
Computer administration: The test is done electronically and thus results are available immediately and can be stored easily. It is also possible to deliver the test over the internet.

\section{Disadvantages of CAT}
\emph{Over usage of a some items: }This issue greatly depends on the way we use to select subsequent questions for students. Nevertheless, with most commonly used criteria there is a danger of selecting the same questions for groups of students and/or selecting certain questions in many tests. For example, the first question, if the selection process is not modified, will be the same for each student. We have no information about the student so far and the selection process results in the same question. Following questions will be the same for groups of students. These groups shrink with more answered questions as the number of possible combinations of answers increase. This behavior can be reduced by having a large question bank containing many different questions with similar properties (i.e. difficulty). Moreover, it is possible to modify the selection process to ensure a wider spread of selected questions over the question bank (with the cost of decreased precision).

\emph{Initial data collection:} Prior to starting a test using CAT it is necessary to obtain a large set of data (full test results) from a representative population. This data is used to create and calibrate the student model used for testing. Results used for this creation need to come from a full length tests (optionally it would be possible, but not preferable, to have a several sub-tests). This means students participating in the initial testing are required to fill answers to many items.

\emph{Building and learning the model: }Before the actual testing starts it is necessary to transform collected data into the student model. This procedure requires certain skill and creates overhead work. 

\emph{Computer administration:} In order to test students it is necessary to create an environment for such testing on the computer. Also it is necessary for students to have access to a computer rather than having just a pen.

\emph{Results perception:} Last but not least, there might be some issues with the perception of results by students taking the test. It may be hard to explain to them and for them to comprehend the fact, that even though they got completely (or partly) different questions they are sorted on the same scale (sometimes even obtaining the same score). It may seem unfair and incomparable because of the question selection process. The feeling may be the same as with the the Czech driving license test mention in the introduction but there the selection is done at random. In reality CAT tests tend to be more fair the regular paper-pen tests~\cite{Moe1988, Tonidandel2002}.
