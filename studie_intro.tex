\chapter*{Introduction} \addcontentsline{toc}{chapter}{Introduction} 
Educational testing is an important part of our lives in the modern society. Every person participates in a large number of tests which are used to assess his/her level of knowledge, quality, or skill in a certain domain. There are many different possibilities how to design a test for a specific purpose. The theory of test creation, administration, validation, etc. (in general called psychometrics) is extensive (for example,~\cite{1964psychometrics, Lord, Rasch1981, Mislevy1994}, and many others). The process of the creation of a good test is long, contains many steps, and can be performed in many different ways. In the classical approach we first identify the target ability (the ability we want to measure). Afterwards, there are repeated cycles of adding new questions, testing the test on a small set of examinees (to see if the questions are measuring the right ability in a correct way), and removing unsuitable questions. In the end we end up with one satisfactory version of the test. This test is fixed in its questions (i.e. every student taking the test will have the same questions). This approach does not take into account the individuality of each examinee. It is clear that for a skilled examinee the test will necessarily contain a lot of questions which are too easy and vice versa. The time which is being spent by solving these questions could be used to better differentiate his/her skill. This can be achieved by asking questions with an appropriate difficulty.
 
There are banks of questions which are suitable to measure the ability of a student (sometimes they are quite limited – for example if we measure a physical ability there might be a limited number of possible questions – and sometimes they are unlimited – in mathematics we can create as many different problems to solve as we want to). Questions for a test are selected from this bank. If we use one set of questions for every test there will be a lot of possibly good questions which are never asked (those which remained in the question bank unselected). The same set of questions for every test also, in some cases, encourages cheating (which of course can usually be solved by other methods, but it requires additional steps). Some tests do not follow the outline explained in the paragraph above and tries to utilize the whole bank of questions. One way of doing that is for example used in the Czech driving license test~\cite{Dopravy2006}. It is a computer test where 25 questions are randomly selected from approximately a thousand of possible questions. This approach negates the possibility of learning all the questions by heart as well as cheating by looking into your neighbor’s sheet. A test in a similar manner is done by a Faculty of Medicine of the Charles University~\cite{UK} as an entrance exam test. Questions for the entrance exam are selected from a set (book) of possible questions for each test (i.e., question bank). Several versions of a test are prepared for every entrance exam session. Both of these selection processes (driver’s license and entrance exam) remove some complications mentioned above but produce new problems. In the driver’s license test, where the question selection is done automatically by the system, it is hard to ensure the overall difficulty of each test will be approximately the same. Cases where a lot of easy questions or on the contrary a lot of difficult questions is selected might occur. In the approach of medical faculty this unfair combination can be avoided by a careful test composition. The test composition is done by hand by specialists but these specialists sometimes have shifted notion of the difficulty of individual questions (some things they may think of as very easy are actually hard for young students). There is also definitely a lot of effort and time involved in the preparation of every entrance exam round.

Computerized Adaptive Testing (CAT) offers a way to overcome some of the limitations given by the classical testing approach. The examinee is answering questions presented to him/her by a computer system. This system is centered on a student model. There are many ways to construct a student model. One way is a model composition by experts. Another is to construct the model from a data set of many previously tested examinees. These examinees have to be tested without the adaptive approach to obtain a basis for the model creation. Afterwards, the model can be further updated and extended with new cases even while being in use. During the course of testing the student model is updated to reflect abilities of the tested student and as a part of that process an estimation of student’s level of knowledge is updated as well. This provides us an actual estimation of student’s abilities in every phase of testing. At the same time the model is used to select a next question. The next selected question is the most appropriate one. An appropriate question suits certain criteria, usually providing the best information about the student at the current stage of testing. Questions are selected from a bank of questions. This bank can be similar to an question bank for the classical test. Adaptive testing is performed until a criterion is reached. There is a variety of possible criteria; usually we want to stop the test when the confidence of the estimation of student’s skill is above a certain significant value. Other practical limitations might affect this criterion such as the total time of a test or the number of asked questions. The adaptive testing concept brings many advantages but also some disadvantages over the classical testing approach. These aspects are detailed in the following chapters. Further we present different model types and perform experiments with our empiric data.
